\documentclass[a4paper]{article}
\usepackage[utf8]{inputenc}

\usepackage[spanish, es-tabla, es-nodecimaldot]{babel}

\usepackage[total={6in, 9in}]{geometry}
\usepackage{amsmath}
%\usepackage{amsfonts}
%\usepackage{amssymb}
\usepackage{graphicx}

\usepackage{hyperref}



\begin{document}

\section{Remuestreo} \label{sec:remuestreo}

En esta secci\'on, analizaremos los efectos del muestreo instant\'aneo en una se\~nal AM. La misma es de la forma:

\begin{equation}
	X_C(t) = \frac{\mathrm{A}_\mathrm{MAX}}{2} \cdot 
	\left(  \frac{1}{2} \cdot \cos{\left( 2\pi \cdot(f_p-f_m) \cdot t\right)} +
	\cos{\left( 2\pi f_p\cdot t\right)} +
	\frac{1}{2} \cdot \cos{\left( 2\pi \cdot (f_p+f_m)\cdot t\right)}
	\right)
\end{equation}

En este caso, se utiliz\'o $f_p=1$kHz y $f_m=100$Hz.  

Para lograr el muestreo instant\'aneo, primero se pasa la se\~nal por el sample and hold. Luego, se la vuelve a muestrear, pero esta vez con la llave anal\'ogica, de manera tal que a la salida se anule la totalidad del tiempo de sample y se conserve s\'olo el de hold. Idealmente, esto es equivalente a multiplicar la se\~nal por un tren de pulsos, cuyo duty cycle coincide con el de la llave anal\'ogica. 
\footnote{Dependiendo de la bibliograf\'ia, puede encontrarse que a esto se refiere como muestreo ``flat top", mientras que con muestreo instant\'aneo se considera al caso de producto con tren de deltas.}



\end{document}