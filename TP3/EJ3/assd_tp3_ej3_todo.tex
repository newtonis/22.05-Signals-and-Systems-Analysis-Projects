\documentclass[a4paper]{article}

\usepackage[utf8]{inputenc}
\usepackage[spanish, es-nodecimaldot]{babel}

\usepackage[total={6in, 9in}]{geometry}

\usepackage{amsmath}
\usepackage{amsfonts}
\usepackage{amssymb}

\renewcommand{\labelitemi}{$\bullet$}
\renewcommand{\labelitemii}{$\circ$}


\date{}
\author{}
\title{ASSD TP3 Modulador Delta - todo list}

\begin{document}
\maketitle

\begin{equation}
	f_s \in [6\text{Hz}, 44.1\text{kHz}]
\end{equation}



\begin{itemize}

	\item Verificaci\'on del funcionamiento - Parte A:
	$f_{CLK} = 44.1$kHz, 
	y entrada rampa de 400Hz con la m\'axima amplitud que admite el sistema.

	\begin{itemize}
		\item Ajuste de la continua de la entrada y la simetr\'ia de la salida
		\item $f_s$ te\'orica y experimental
		\item Medir el error de cuantizaci\'on, con la rampa u otra se\~nal
		\item Determinar SNR m\'axima para se\~nales de AUDIO
		\item ?`Cu\'al  es el ENOB (effective number of bits) del sistema? 
				?`C\'omo se midi\'o? 
		\item Medir con distintas cantidad de bits activos
	\end{itemize}

	
	\item Verificaci\'on de funcionamiento - Parte B:
	$f_{CLK} = 15$kHz, entrada sinusoidal de 400Hz entre 2V y 2.05V. 
	Observar la salida, el espectro, la distribuci\'on de potencia del ruido de 
	cuantizaci\'on.
	
	
	\item Verificaci\'on de funcionamiento - Parte C:
	Entrada continua, medir entrada, salida y ver qu\'e n\'umero marcan los LEDs.
	\begin{itemize}
		\item Graficar $V_O ( V_I )$. 
		?`Qu\'e errores se pueden inferir de lo obtenido?
		\item Forma de onda del error total de los conversores	
		\item Comparar resultados con la hoja de datos del conversor
	\end{itemize}		

	
	\item Verificaci\'on de funcionamiento - Parte D:
	Con los filtros, tensi\'on de entrada m\'axima y frecuencia de entrada m\'axima, medir 	
	con el analizador de espectros y simular: senoidal, rampa, sinc, m\'usica.  Con 
	$f_s = 80$kHz y despu\'es con $f_s=44.1$kHz.
	
	
	\item Modulador Delta
	\begin{itemize}
		\item ?`C\'omo se demodula?
		\item ?`Por qu\'e hace falta o no usar un SH?
		\item Ciclios de clock para una conversi\'on
		\item ?`Cu\'al es el par\'ametro limitante? Medirlo
	\end{itemize}
\end{itemize}


\end{document}