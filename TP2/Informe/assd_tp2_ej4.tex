\documentclass[assd_tp2_main.tex]{subfiles}

\begin{document}

\section{Síntesis mediante modulación en frecuencia}

La modulación en frecuencia esta dada por:
\begin{eqnarray*}
\textstyle x(t)=A(t)cos(2\pi f_c t+I(t)cos(2\pi f_mt+\phi_m)+\phi_c)
\end{eqnarray*}
Para los intrumentos de viento:
\begin{eqnarray*}
\displaystyle \phi_m=\phi_c=-\frac{\pi}{2}
\end{eqnarray*}
Hay dos consideraciones adicionales a la hora de sintetizar un instrumento de viento:
\begin{itemize}
  \item Para cada instrumento corresponden un A(t) e I(t).

  \item Para cada instrumento hay un factor 
\begin{eqnarray*}
\displaystyle \frac{f_c}{f_m}=\frac{n}{m}=\frac{c}{m}
\end{eqnarray*}

Y también se encuentra $f_0$ (también conocida como note frequency) que resulta ser el máximo común denominador de fc y fm.
\end{itemize}

 
\begin{eqnarray*}
\displaystyle f_0=gcd(f_c,f_m)
\end{eqnarray*}
\subsection{Síntesis de clarinete}
Para un clarinete se utiliza un esquema similar al ADSR con Attack Sustain y Release. En lugar de tratarse de una interpolación lineal entre puntos, se trata de una exponencial.
\\
Para una campana se suele utilizar

\end{document}

