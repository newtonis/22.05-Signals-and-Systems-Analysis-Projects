\documentclass[assd_guia_filtros_recursivos_main.tex]{subfiles}

\begin{document}

\section{Ejercicio 11}

\subsection{Sistema 1}

Se resolvió mediante transformada Z. Se obtuvó

\begin{equation}
G(z)=X(z)H(z)
\end{equation}
\begin{equation}
R(z)=G(1/z)H(z)
\end{equation}
Por lo tanto
\begin{equation}
Y(z)=Y(1/z)=G(z)H(1/z)=X(z)H(z)H(1/z)
\end{equation}
Entonces
\begin{equation}
H_2(n)=H(n)*H(-n)
H_2(z)=H(z)H(1/z)
\end{equation}
Calculando la amplitud de $H_2(f)=H_2(e^{j2\pi f})$
\begin{equation}
|H_2(f)|^2=H_2(z)H_2(1/z)=H(z)H(1/Z)H(z)H(1/z)=|H(z)|^4=|H(f)|^4
\end{equation}
Y la fase (asumiendo que $h(n)$ es real)
\begin{equation}
\angle H_2(f)= \angle H(e^{i2\pi f}) + \angle H(e^{-i2\pi f}) = 0
\end{equation}

\subsection{Sistema 2}

Se resolvió nuevamente mediante transformada Z. Se obtuvó
\begin{equation}
G(z)=X(z)H(z)
\end{equation}
\begin{equation}
R(z)=X(1/z)H(z)
\end{equation}
\begin{equation}
Y(z)=G(z)+R(1/z)=X(z)H(z)+X(z)H(1/z)=X(z)( H(z) + H(1/z)
\end{equation}
Por lo tanto
\begin{equation}
H_2(z)=H(z)+H(1/z)
\end{equation}
\begin{equation}
H_2(n)=H(n)+H(-n)
\end{equation}
Calculando módulo y fase cuando $z=e^{jw}$ asumiendo $x(n)$ real

\begin{equation}
H_2(z)=H(z)+H(1/z)=H(f)+H(-f)=H(f)+H^{*}(f)=2Re(H(f))
\end{equation}
Por lo tanto
\begin{equation}
|H_2(f)|=2|Re(H(f))|
\end{equation}
\begin{equation}
\angle H_2(f)=0
\end{equation}

No se llegó a resolver la segunda parte del ejercicio 11

\end{document}